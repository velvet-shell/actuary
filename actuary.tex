\documentclass[10pt]{beamer}
\usepackage[T2A]{fontenc}
\usepackage[utf8]{inputenc}
\usepackage[english,russian]{babel}
\usepackage{amssymb,amsfonts,amsmath,mathtext}
\usepackage{bbm}
\usepackage{setspace}
\usepackage{lipsum}
\usepackage{cite,enumerate,float}
\usepackage{pgfplots}
\pgfplotsset{compat=newest}

\setlength{\parindent}{3ex}
\setlength{\parskip}{0.5em}

\usetheme{Pittsburgh}
\usecolortheme{whale}

\usefonttheme[onlymath]{serif}

\newcommand\Exp{{\rm exp \,}}
\newcommand\E{\mathbb{E}}
\newcommand\D{\mathbb{D}}
\newcommand\Pro{\mathbb{P}}

\counterwithin*{equation}{section}
\counterwithin*{equation}{subsection}

\title[Модель коллективного риска]{Вероятность разорения в классической модели коллективного риска} 
\author{Кодзоев М., Куркин М., Шаповалов Р.}
\institute[ВМК МГУ]
{
Московский Государственный Университет им. Ломоносова
}
\date{\today}

\begin{document}
\begin{frame}       
\titlepage
\end{frame}


\begin{frame}
\tableofcontents
\end{frame}


\section{Описание модели}
\begin{frame}
\frametitle{Модель с непрерывным временем}
\begin{flushleft}
Математическая модель изменения величины рискового резерва на длительном интервале времени:
\end{flushleft}

$U(t) = u + c(t) - S(t),\quad t \geq 0$
\begin{itemize}
    \item $U(t)$ - рисковый резерв в момент времени $t$,
    \item $u$ - величина рискового резерва в момент времени 0,
    \item $c(t)$ - величина премий, собранных к моменту $t$,
    \item $S(t)$ - величина суммарных страховых выплат до момента $t$.
\end{itemize}
\end{frame}


\begin{frame}
\frametitle{Модель с непрерывным временем}
\noindent
Рассмотрим величину суммарных страховых выплат до момента $t$:
\par\medskip
$S(t) = X_{1}+X_{2}+...+X_{N(t)}$
\begin{itemize}
    \item $\{S(t), t \geq 0\}$ - процесс суммарных выплат,
    \item $\{N(t), t \geq 0\}$ - процесс числа страховых случаев,
    \item $X_i$ - величина $i$-ой страховой выплаты.
\end{itemize}
\par\medskip \noindent
Пусть $t\geq 0$ и $h> 0$ . Тогда разность $N(t+h)-N(t)$ является числом страховых случаев,
а разность $S(t+h) -S(t)$ - суммарными страховыми выплатами,
которые происходят в интервале между $t$ и $t + h$.
\end{frame}


\begin{frame}
\frametitle{Модель с непрерывным временем}
\noindent
Для моделирования процесса числа страховых случаев $\{N(t), t \geq 0\}$ воспользуемся
пуассоновским процессом, для которого длины интервала времени между последовательными страховыми
случаями являются взаимно независимыми и одинаково распределенными случайными величинами
с показательным распределением.
\par\medskip \noindent
\begin{equation*}
\mathbb{P}[N(t+h)-N(t)=k] = \frac{e^{- \lambda h}(\lambda h)^{k}}{k!}
\;\; \forall t \geq 0, h > 0, k = 0, 1, 2, \dots
\end{equation*}
\end{frame}


\begin{frame}
\frametitle{Модель с непрерывным временем}
\noindent
Из этого определения вытекают следующие свойства:
\begin{enumerate}
    \item[1)] Приращения стационарны, то есть распределение $N(t+h)-N(t)$ не зависит от $t$;
    \item[2)] Для любого множества непересекающихся временных интервалов приращения независимы,
    то есть для
    \par\smallskip
    $t_{1}<t_{1}+h_{1}<t_{2}<t_{2}+h_{2}<...<t_{n}+h_{n}$
    приращения:
    \par\smallskip
    $N(t_{1}+h_{1})-N(t_{1}), N(t_{2}+h_{2})-N(t_{2}),...\:,N(t_{n}+h_{n})-N(t_{n})$
    \par\smallskip
    взаимно независимы;
    \item[3)] Вероятность того, что несколько страховых случаев произойдет одновременно, равна нулю,
    то есть:
    \begin{equation*}
        \lim_{h \rightarrow 0} \: \frac{\mathbb{P}[N(t+h)-N(t)>1]}{h} =
        \lim_{h \rightarrow 0} \: \frac{1-e^{- \lambda h}- \lambda he^{- \lambda h}}{h} = 0
    \end{equation*}
\end{enumerate}
\end{frame}


\begin{frame}
\frametitle{Модель с непрерывным временем}
\noindent
Если в $S(t)$ случайные величины $X_{1}, X_{2}, X_{3}, ...$ независимы и одинаково распределены
с функцией распределения $P(x)$, и если они также независимы от процесса
$\big\{N(t), t\geq 0\big\}$, то процесс $\big\{S(t), t\geq 0\big\}$ называется
\textbf{сложным пуассоновским процессом}.
\begin{block} {Свойства:}
\begin{enumerate}
    \item[1)] Если $t\geq 0 $ и $h > 0$, то распределение c.в. $S(t + h) - S(t)$
    является сложным пуассоновским c параметром $\lambda h$ и функцией распределения $P(x)$, то есть:
    \begin{equation*}
        \mathbb{P}[S(t+h)-S(t)\leq x] = 
        \sum_{k=0}^{ \infty }e^{- \lambda h}(\lambda h)^{k} \frac{P^{\ast k}(x) }{k!}
    \end{equation*}
    \item[2)] В любой момент $t$ вероятность того, что следующий страховой случай произойдет
    между моментами $t + h$ и $t + h + dh$ и что величина выплат не превосходит $x$, равна
    $e^{-\lambda h}(\lambda dh)P(x)$;
    \item[2)] Приращения процесса $S(t)$ независимы и стационарны;
    \item[3)] $\mathbb{E}[S(t)] = \lambda t p_1,\ \mathbb{D}[S(t)] = \lambda t p_2$.
\end{enumerate}
\end{block}
\end{frame}


\section{Коэффициент Лундберга и теорема о вероятности разорения}
\begin{frame}
\frametitle{Некоторые понятия}
\begin{block}{Определение 1}
Рассмотрим период длины $t > 0$, где размер собранной  премии равен $ct$, а $S(t)$
- величина суммарных страховых выплат, распределение которой - сложное пуассоновское,
причём $\mathbb{E}N(t) = {\lambda}t$.
\\ \textbf{Коэффициент Лундберга ${R}$} - наименьшее положительное решение уравнения
$$M_{S(t)-ct}(r) = \mathbb{E}[e^{r(S(t)-ct)}] = e^{-rct}M_{S(t)}(r) = $$
$$= e^{-rct}e^{{\lambda}t[M_{X}(r)-1]} = 1 \Leftrightarrow {\lambda}[M_{X}(r)-1] = cr,$$
или, при $c = (1+{\theta}){\lambda}p_{1}r$, уравнения 
\begin{center}$1+(1+\theta)p_{1}r = M_{X}(r)$ \textbf{(2.1)} \end{center}
\end{block}
\begin{block}{Определение 2}
\textbf{Вероятность разорения $\psi(u)$} равна $\mathbb{P}(T < \infty)$,
где $T = min\{t: U(t) < 0\}$. Она выражается с помощью коэффициента Лундберга.
\end{block}
\end{frame}


\begin{frame}
\begin{center}
    \begin{tikzpicture}
        \begin{axis}[
            axis lines = left,
            xlabel = {$r$},
            ylabel = {}, 
            every axis x label/.style={at={(current axis.right of origin)},anchor=west},
            ticks = none,
        ]
        \addplot[
            color=red, 
            domain = 0:2,
        ]
        {x+1};
        \addplot[
            color=blue,
            domain = 0:2,
        ]
        {x^(3/2)+1};
        \addplot[dashed] coordinates { (1.0, 2.0) (1.0, 0) };
        \draw (1.5,2.08) node {$1+(1+\theta)p_{1}r$};
        \draw (1.5,3.2) node {$M_{x}(r)$};
        \draw (1.1,0.15) node {$R$};
        \end{axis}
    \end{tikzpicture}

    Определение коэффициента Лундберга $R$
\end{center}
\end{frame}

\begin{frame}
\frametitle{Некоторые понятия}
\begin{block}{Теорема о вероятности разорения}
Если $U(t)$ является процессом рискового резерва, его процесс суммарных страховых выплат $S(t)$
является сложным пуассоновским и если $c > \lambda p_{1}$, т. е. рисковая надбавка положительна,
то для $u \geq 0$ справедливо 
$$ \psi(u) = \frac{\Exp(-Ru)}{\mathbb{E}[\Exp(-RU(T))|T < \infty]} \: \textbf{(2.2)}, $$
где $R$ - наименьший положительный корень уравнения \textbf{(2.1)}.
\end{block}
\end{frame}


\section{Вычисления}
\subsection{Коэффициент Лундберга}
\begin{frame}
\frametitle{Вычисление вероятности разорения}
\noindent
Рассмотрим случай с показательным распределением величины страховых выплат с параметром $\beta > 0$.

\noindent
\textbf{1) Определим коэффициент Лундберга:}

\noindent
Уравнение (2.1) принимает вид
\begin{equation}
1 + \frac{(1 + \theta)r}{\beta}=\frac{\beta}{\beta-r}
\end{equation}
или в форме квадратного уравнения по $r$,
\begin{equation}
(1 + \theta)r^2 - \theta\beta r = 0
\end{equation}
Положительное решение уравнения является коэф. Лундберга:
\begin{equation}
R = \frac{\theta\beta}{1 + \theta}
\end{equation}
\end{frame}

\subsection{Вероятность разорения}
\begin{frame}
\frametitle{Вычисление вероятности разорения}
\noindent
\textbf{2) Вычислим вероятность разорения:}

\noindent
Пусть разорение, если оно происходит, случается в момент $T$.
Пусть $\hat{u}$ является величиной рискового резерва непосредственно перед моментом $T$.
\begin{equation}
\mathbb{P}(-U(T)>y\ |\ T < \infty) = \mathbb{P}(X > \hat{u} + y \ | \ X > \hat{u}) =
\frac   {\beta\int_{\hat{u}+y}^{\infty} e^{-\beta x}dx}
        {\beta\int_{\hat{u}}^{\infty}   e^{-\beta x}dx}
                                                        = e^{-\beta y}
\end{equation}

\begin{equation}
p_{-U(T)|T < \infty}(y) = \frac{d}{dy}(1-e^{-\beta y}) = \beta e^{-\beta y} \mathbbm{1}(y > 0)
\end{equation}

\begin{equation}
\mathbb{E}[e^{-RU(T)} \ | \ T<\infty] = \beta\int_{0}^{\infty} e^{-\beta y} e^{Ry}dy =
\frac{\beta}{\beta - R}
\end{equation}
\end{frame}


\begin{frame}
\frametitle{Вычисление вероятности разорения}

\noindent
Воспользуемся вычисленным ранее коэффициентом Лундберга и теоремой:
\begin{equation}
\psi(u) = \frac{e^{-Ru}}{\mathbb{E}[e^{-RU(T)}|T < \infty]} = \frac{(\beta-R)e^{-Ru}}{\beta}
\end{equation}

\begin{equation}
R = \frac{\theta\beta}{1 + \theta} \Rightarrow
\psi(u) = \frac{1}{1+\theta} \Exp\left(\frac{-\theta\beta u}{1+\theta}\right)
\end{equation}
\end{frame}


\section{Приложение}
\subsection{Дополнительные выкладки}
\begin{frame}
\frametitle{Приложение}
\begin{itemize}
    \item Математическое ожидание и дисперсия сложного пуассоновского процесса:
    \begin{align}
    & \E[S] = \E\big[\E[S|N]\big]=\E[p_1 N] = p_1\E[N] = \lambda t p_1 \\
    & \D[S] = \E\big[\D[S|N]\big] + \D\big[\E[S|N]\big] = \E\big[N\D[X]\big] + \D[p_1N] =\notag \\
    & = \E[N]\D[X] + p_1 ^2 \D[N] = \lambda t (p_2 - p_1^2) + p_1^2 \lambda t = \lambda t p_2
    \end{align}
    \item Производящая функция моментов сложного пуассоновского процесса:
    \begin{align}
    & M_S (r) = \mathbb{E}[e^{rS}]=\mathbb{E}\big[\mathbb{E}[e^{rS}|N]\big]=\mathbb{E}[M_X (r)^N]=
    \notag \\
    & = \mathbb{E}[e^{N \ln{M_X (r)}}]= M_N (\ln{M_X (r)}) \\
    & M_N (r) = e^{\lambda t(e^r - 1)} \Rightarrow M_S (r) = e^{\lambda t[M_X (r) - 1]}
    \end{align}
\end{itemize}
\end{frame}


\subsection{Доказательство теоремы}
\begin{frame}
\frametitle{Доказательство теоремы}
\noindent
Рассмотрим доказательство теоремы о вероятности разорения:
\begin{equation}
\E[e^{-rU(t)}] = \E[e^{-rU(t)}|T \leq t]\Pro(T \leq t) + \E[e^{-rU(t)}|T > t]\Pro(T > t)
\end{equation}

\noindent
Левая часть равна: $\Exp\{-ru - rct + \lambda t [M_X(r) - 1]\}$.
В первом слагаемом в правой части:
\begin{equation}
U(t) = U(T) + [U(t) - U(T)] =  U(T) + c(t - T) - [S(t) - S(T)]
\end{equation}

\noindent
$S(t) - S(T)$ имеет сложное распределение Пуассона с $\lambda (t -T)$.
\begin{equation}
\E\big[\Exp\big(-rU(T)\big)\Exp\big(-rc(t-T) + \lambda (t - T)[M_x(r) - 1]\big)|T \leq t\big] 
\Pro (T \leq t)
\end{equation}

\noindent
Упростим выражения, выбрав r так, что: $-rc + \lambda [M_X(r) - 1] = 0$.
Подставим полученные выражения c $r = R > 0$ в исходное:
\begin{equation}
e^{-Ru} = \E[e^{-RU(T)}|T \leq t]\Pro(T \leq t) + \E[e^{-RU(t)}|T > t]\Pro(T > t)
\end{equation}
\end{frame}


\begin{frame}
\frametitle{Доказательство теоремы}
\begin{equation}
e^{-Ru} = \E[e^{-RU(T)}|T \leq t]\Pro(T \leq t) + \E[e^{-RU(t)}|T > t]\Pro(T > t)
\end{equation}
Пусть $t \rightarrow \infty$. Первое слагаемое в правой части сходится к
\begin{equation}
\E[e^{-RU(T)}|T < \infty]\psi(u)
\end{equation}
\noindent
Для доказательства теоремы остается доказать сходимость второго слагаемого к $0$,
используя неравенство Чебышева, после чего получается требуемое равенство:
\begin{equation}
\psi(u) = \frac{e^{-Ru}}{\mathbb{E}[e^{-RU(T)}|T < \infty]}
\end{equation}
\end{frame}


\begin{frame}
\frametitle{Использованная литература}
\footnotesize{
\begin{thebibliography}{99}
\bibitem[Н. Бауэрс, Х. Гербер, Д. Джонс, С. Несбитт, Дж. Хикман 1997]{p1} Н. Бауэрс, Х. Гербер,
Д. Джонс, С. Несбитт, Дж. Хикман (1997)
\newblock \emph{Актуарная математика}, 355 -- 368.
\end{thebibliography}
}
\end{frame}
\end{document}